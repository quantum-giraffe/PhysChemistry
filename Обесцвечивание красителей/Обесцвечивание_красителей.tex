\documentclass[a4paper, 12pt]{article}
\usepackage[T2A,T1]{fontenc}
\usepackage[utf8]{inputenc}
\usepackage[english, russian]{babel}
\usepackage{graphicx}
\usepackage[hcentering, bindingoffset = 10mm, right = 13 mm, left = 13 mm, top=20mm, bottom = 20 mm]{geometry}
\usepackage{multirow}
\usepackage{lipsum}
\usepackage{amsmath, amstext}
\usepackage{siunitx}
\usepackage{subcaption}
\usepackage{wrapfig}
\usepackage{mathrsfs}
\usepackage{adjustbox}
\usepackage{enumerate, indentfirst, float}
\usepackage{capt-of, svg}
\usepackage{icomma}
\usepackage{xcolor}
\usepackage{ctable}
\usepackage{multirow}

\newenvironment{bottompar}{\par\vspace*{\fill}}{\clearpage}
 
\begin{document}
\begin{titlepage}

\newcommand{\HRule}{\rule{\linewidth}{0.5mm}} % Defines a new command for the horizontal lines, change thickness here

\center % Center everything on the page
 
%----------------------------------------------------------------------------------------
%	HEADING SECTIONS
%----------------------------------------------------------------------------------------

\textsc{\LARGE Московский Физико-Технический Институт}\\[1,5cm] % Name of your university/college
\textsc{\Large Департамент молекулярной и биологической физики}\\[2cm] % Major heading such as course name
\textsc{\large Лабораторная работа}\\[0.5cm] % Minor heading such as course title

%----------------------------------------------------------------------------------------
%	TITLE SECTION
%----------------------------------------------------------------------------------------

\HRule
\\[0.4cm]
{ \huge \bfseries Кинетика обесцвечивания красителя \\в щелочной среде}
\\[0.2cm] % Title of your document
\HRule
\\[1.5cm]


 
%----------------------------------------------------------------------------------------
%	AUTHOR SECTION
%----------------------------------------------------------------------------------------
\begin{minipage}{0.4\textwidth}
	\begin{flushleft}		
	\end{flushleft}
\end{minipage}
~
\begin{minipage}{0.4\textwidth}
	\begin{flushright} \large
		\emph{Авторы:}\\
		Светлана \textsc{Фролова} \\
		6113 группа \\
		Анатолий \textsc{Киселёв} \\
		6113 группа
	\end{flushright}
\end{minipage}


\begin{bottompar}
	\begin{center}
		\includegraphics[width = 80 mm]{logo.jpg}
	\end{center}
	{\large \today}

\end{bottompar}
\vfill % Fill the rest of the page with whitespace

\end{titlepage}

\newpage
\section{Цели работы}
	\begin{enumerate}
	
		\item Определить порядок реакции по фенолфталеину и по NaOH;
		
		\item Рассчитать константу скорости реакции;
		
		\item Исследовать
влияние ионной силы раствора на скорость реакции.
	
	\end{enumerate}
	
\section{Теоретическая часть}
	\subsection*{Определение константы скорости из спектрофотометрических измерений}
	Скорость реакции будет определяться на основе спектрофотометрических
измерений, позволяющих зафиксировать изменение концентрации красителя в растворе. Закон светопоглощения Бугера-Ламберта-Бера устанавливает прямую
пропорциональную зависимость между концентрацией поглощающего вещества $C$ и
измеряемой оптической плотностью при заданной длине волны излучения $\lambda$:


\[D=\lg{\frac{I}{I_0}}=\varepsilon_\lambda Cl,\]
где $D$ - оптическая плотность; $I_0$ – интенсивность падающего света; $I$ – интенсивность
света, прошедшего через образец; $C$ – концентрация вещества; $l$ – толщина
поглощающего слоя; $\varepsilon_\lambda$ – коэффициент молярной экстинкции при длине волны $\lambda$. В
данном случае он совпадает с коэффициентом поглощения, хотя в общем случае
учитывается также процесс ослабления света из-за рассеяния $\varepsilon_\lambda = \varepsilon_{\lambda,abs} + \varepsilon_{\lambda,scatt}.$
Для реакции первого или псевдопервого порядка константа скорости может быть
записана в виде $k_1=\frac{1}{t}\cdot\ln{\frac{C_0}{C(t)}}=\frac{1}{t}\cdot\ln{\frac{D_0}{D(t)}},$
где $D_0$ – начальная величина оптической
плотности, $D(t)$ – ее значение к моменту времени $t$.

	\subsection*{Влияние среды на скорость ионных реакций}
	Реакции между заряженными частицами в растворах сопровождаются рядом
специфических эффектов, обусловленных наличием электростатических взаимодействий ионов друг с другом и со средой, из свойств которой наиболее важными по
их влиянию на состояние и взаимодействие заряженных частиц являются диэлектрическая проницаемость $\varepsilon$ и ионная сила $J$. Последняя представляет собой обобщенную
концентрацию ионов в растворе, записанную в следующей форме:

\[J= \frac{1}{2} \sum{C_i Z_i^2},\]
где $C_i$ и $Z_i$ — концентрации и заряды различных ионов соответственно.
Теория влияния $\varepsilon$ и $J$ на скорости реакций заряженных или полярных частиц в
растворах детально разработаны и широко используются для установления механизма
реакций. Рассмотрим аспекты этой теории, необходимые для понимания явлений,
экспериментально наблюдаемых в данной работе.
В теории переходного состояния или активированного комплекса скорость
химической реакции представляют выражением

 \[w = \frac{k_\text{Б}T}{h}\cdot C^\#,\]
где $Т$ — температура, $k_\text{Б}$ и $h$ — константы Больцмана и Планка, а $C^\#$ — концентрация активированного комплекса. Множитель $(k_\text{Б}T / h)$ представляет собой универсальную, т.е. независящую от природы реагентов частоту перехода комплексов через вершину
активационного барьера.
Поскольку в данной теории постулируется существование равновесия между
исходными реагентами и активированным комплексом, концентрацию последнего
можно получить, используя, хотя и формально, имеющийся аппарат расчета констант
равновесия $K_a^\#$. В частности, для бимолекулярной реакции с участием заряженных частиц A и B в растворе она записывается в следующей форме:
\[K_a^\#=\frac{a^\#}{a_Aa_B}=\frac{C^\#}{C_AC_B}\cdot\frac{f^\#}{f_Af_B},\]
в которой $a$ и $f$ представляют собой активности и коэффициенты активности реагентов
и активированного комплекса ($a_i = f_i C_i$), концентрация которого составит:

\[C^\#=K_a^\#\frac{f^\#}{f_Af_B}Ca_AC_B.\]
Отсюда получим выражение для константы скорости реакции
\[k=\frac{w}{C_AC_B}=\frac{k_\text{Б}T}{h}K_a^\#\cdot\frac{f^\#}{f_Af_B}=k_0\frac{f^\#}{f_Af_B},\]
где в постоянную величину $k_0$ включены все независимые от свойств среды параметры.
Что касается коэффициентов активности ионов, то в теории сильных электролитов
Дебая-Хюккеля они определяются уравнением

\[-\lg{f_i}=1,823\cdot10^6(\varepsilon T)^{-\frac{3}{2}}Z_i^2\sqrt{J}.\]
Преобразуя комбинацию коэффициентов активности в этом выражении с учетом $Z^\#=Z_A+Z_B$, получим окончательно:

\[\lg{\left(\frac{k}{k_0}\right)}\simeq 1,02\cdot Z_AZ_B\sqrt{J}.\]
Анализ полученного выражения показывает, что в случае одноименно
заряженных ионов ($Z_AZ_B > 0$) константа скорости реакции растет с ионной силой, тогда
как в случае разноименных ($Z_AZ_B < 0$) – уменьшается. Зависимость скорости реакций в
растворах от ионной силы носит название первичного солевого эффекта, причем знак и
масштаб влияния $J$ на $w$ позволяют установить зарядность частиц, участвующих в
лимитирующей стадии процесса, по наклону графиков $\lg{\left(k/k_0\right)}$ от $\sqrt{J}$.

\newpage
\section{Обработка результатов}


\begin{figure}[h]
	\centering
	\caption{$D=f(t)|_{J=const}$}
	\includegraphics[width=0.8\textwidth]{Figure_1.png}
\end{figure}

\begin{figure}[b!]
	\centering
	\caption{$D=f(t)|_{C=const}$}
	\includegraphics[width=0.8\textwidth]{Figure_2.png}
\end{figure}

\newpage
\section*{$J=const$}

\begin{figure}[h!]
	\centering
	\caption{}
	\includegraphics[width=0.8\textwidth]{image003.png}
\end{figure}

\begin{figure}[h!]
	\centering
	\caption{}
	\includegraphics[width=0.8\textwidth]{image005.png}
\end{figure}

Порядок реакции по малахитовому зеленому: $m=1$;
по $[OH^-]$: $n=1$

Константа скорости $k_2=0.462$



\newpage
\section*{$C=const$}

\begin{figure}[h!]
	\centering
	\caption{}
	\includegraphics[width=0.8\textwidth]{1.jpg}
\end{figure}

\begin{figure}[h!]
	\centering
	\caption{}
	\includegraphics[width=0.8\textwidth]{image006.png}
\end{figure}

$Z_AZ_B<0 \Rightarrow $ частицы разноименны\\


$k_2=1,02\cdot Z_AZ_B=-0,60$

\newpage
\section{Вывод}
Мы изучили спектрофотометрический метод нахождения зависимости константы реакции от концентрации веществ и ионной силы раствора.

\end{document}