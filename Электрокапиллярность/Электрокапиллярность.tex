\documentclass[a4paper, 12pt]{article}
\usepackage[T2A,T1]{fontenc}
\usepackage[utf8]{inputenc}
\usepackage[english, russian]{babel}
\usepackage{graphicx}
\usepackage[hcentering, bindingoffset = 10mm, right = 13 mm, left = 13 mm, top=20mm, bottom = 20 mm]{geometry}
\usepackage{multirow}
\usepackage{lipsum}
\usepackage{amsmath, amstext}
\usepackage{siunitx}
\usepackage{subcaption}
\usepackage{wrapfig}
\usepackage{mathrsfs}
\usepackage{adjustbox}
\usepackage{enumerate, indentfirst, float}
\usepackage{capt-of, svg}
\usepackage{icomma}
\usepackage{xcolor}
\usepackage{ctable}
\usepackage{amssymb}
\usepackage[version=4]{mhchem}
\usepackage{expl3}
\usepackage{calc}


\newenvironment{bottompar}{\par\vspace*{\fill}}{\clearpage}
 
\begin{document}
\begin{titlepage}

\newcommand{\HRule}{\rule{\linewidth}{0.5mm}} % Defines a new command for the horizontal lines, change thickness here

\center % Center everything on the page
 
%----------------------------------------------------------------------------------------
%	HEADING SECTIONS
%----------------------------------------------------------------------------------------

\textsc{\LARGE Московский физико-технический институт\\(государственный университет)}\\[1,5cm] % Name of your university/college
\textsc{\Large Департамент молекулярной и биологической физики}\\[2cm] % Major heading such as course name
\textsc{\large Лабораторная работа}\\[0.5cm] % Minor heading such as course title

%----------------------------------------------------------------------------------------
%	TITLE SECTION
%----------------------------------------------------------------------------------------

\HRule
\\[0.2cm]
{ \huge \bfseries Электрокапиллярные явления.\\Свойства электродов}
\\[0.2cm] % Title of your document
\HRule
\\[1.5cm]


 
%----------------------------------------------------------------------------------------
%	AUTHOR SECTION
%----------------------------------------------------------------------------------------
\begin{minipage}{0.4\textwidth}
	\begin{flushleft}		
	\end{flushleft}
\end{minipage}
~
\begin{minipage}{0.4\textwidth}
	\begin{flushright} \large
		\emph{Авторы:}\\
		Светлана \textsc{Фролова} \\
		6113 группа \\
		Анатолий \textsc{Киселёв} \\
		6113 группа
	\end{flushright}
\end{minipage}


\begin{bottompar}
	\begin{center}
		\includegraphics[width = 80 mm]{logo.jpg}
	\end{center}
	{\large г. Долгопрудный\\2018 г.}

\end{bottompar}
\vfill % Fill the rest of the page with whitespace

\end{titlepage}

\setcounter{page}{2}

\newpage
\section{Цели работы}
	\begin{enumerate}
	
		\item 
		Определение зависимости поверхностного натяжения на границе ртуть-раствор
электролита от электрического потенциала.
		\item 
		 Определение потенциала нулевого заряда и емкости двойного электрического
слоя на поверхности ртутного электрода в растворе.
		\item 
		 Исследование влияния природы электролита на потенциал нулевого заряда.
		\item 
		 Получение хлорсеребряного электрода.
		\item 
		 Исследование поляризуемости различных электродов. Выявление электродных
процессов, ограниченных стадией массопереноса и стадией переноса заряда.
			
	\end{enumerate}
	
\section{Теоретическая часть}


\newpage
\section{Обработка результатов}


\begin{figure}[h]
	\centering
	\caption{}
%	\includegraphics[width=1\textwidth]{.png}
\end{figure}
\newpage

\begin{figure}[t]
	\centering
	\caption{}
%	\includegraphics[width=1\textwidth]{.png}
\end{figure}

\begin{figure}[h!]
	\centering
	\caption{}
%	\includegraphics[width=1\textwidth]{.png}
\end{figure}

\newpage
\section{Вывод}
\end{document}